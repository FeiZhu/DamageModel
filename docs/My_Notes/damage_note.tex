\documentclass[11pt,fullpage]{article}
%\usepackage{multicol,wrapfig,amsmath,subfigure}
\usepackage{multicol,amsmath,amssymb,algorithmic,bm}
\usepackage{graphics,graphicx}
\usepackage[right = 2.5cm, left=2.5cm, top = 2.5cm, bottom =2.5cm]{geometry}
\pagestyle{plain}
\def\urlfont{\DeclareFontFamily{OT1}{cmtt}{\hyphenchar\font='057}
              \normalfont\ttfamily \hyphenpenalty=10000}

%\input macros

\newcommand{\subheading}[1]{\noindent \textbf{#1}}
\newcommand{\grad}{\nabla}
\newcommand{\jump}[1]{[#1]}
\newcommand{\limit}[2]{\lim_{#1 \rightarrow #2}}
\newcommand{\mb}[1]{\mathbf{#1}}
\newcommand{\reals}[1]{\mathbb{R}^{#1}}


\begin{document}
\noindent
\begin{center}
{\bf A Damage Model for Meshless Fracture Simulation}
{\\ Fei Zhu \\ 10/28/2014}
\end{center}

\section{Basic Concepts}

\subsection{Damage State}

We represent the damage state in the material with a second-order
tensor $\bm{D}$. It is defined by a linear transformation from the
vector $\bm{n}dA$ in the current damaged configuration into
$\tilde{\bm{n}}d\tilde{A}$ in the fictitious undamaged configuration
that is mechanically equivalent:
$$
\tilde{\bm{n}}d\tilde{A} = (\bm{I}-\bm{D})\bm{n}dA,
$$
where $\bm{n}dA$ and $\tilde{\bm{n}}d\tilde{A}$ are the surface
elements in the two configurations. The damage variable characterizes
the net area reduction caused by damage of material.

\subsection{Effective Stress Tensor and Damage Effect Tensor}

$$
\sigma(\bm{n}dA) = \tilde{\sigma}(\tilde{\bm{n}}d\tilde{A}) 
                 = \tilde{\sigma}(\bm{I}-\bm{D})(\bm{n}dA).
$$
From this, the effective stress tensor $\tilde{\sigma}$ is given by:
$$
\tilde{\sigma} = (\bm{I}-\bm{D})^{-1}\sigma.
$$
The effective stress tensor defined above is asymmetric, it is usually
inconvenient to use it in the formulation of constitutive and
evolution equations. Different schemes of symmetrization have been
proposed, which can be expressed in a unified form as:
$$
\tilde{\sigma} = \mathbb{M}(\bm{D}):\sigma,
$$
where $\mathbb{M}(\bm{D})$ is a fourth-order tensor-valued tensor
function which transforms the stress tensor into corresponding
effective stress tensor, and is called a damage effect tensor.

It is often convenient to express tensors in the form of the matrices
of their components, and the to execute the tensor operation as a
matrix calculus. According to the Voigt notation, the second-order
symmetric stress tensor $\sigma$ and the related effective stress
tensor $\tilde{\sigma}$ are expressed by the column vectors of
six-dimension:
$$
[\sigma] = [\sigma_{11}\ \sigma_{22}\ \sigma_{33}\ \sigma_{23}\
\sigma_{12}\ \sigma_{13}]^T
$$
$$
[\tilde{\sigma}] = [\tilde{\sigma}_{11}\ \tilde{\sigma}_{22}\
\tilde{\sigma}_{33}\ \tilde{\sigma}_{23}\ \tilde{\sigma}_{12}\ \tilde{\sigma}_{13}]^T
$$
Considering the symmetry of stress tensors, the damage effect tensor
is represented as a six by six matrix $[\bm{M}]$. Then:
$$
[\tilde{\sigma}] = [\bm{M}][\sigma].
$$
$[\bm{M}]$ is a
diagonal matrix if the principal directions of the damage tensor
$\bm{D}$ are taken as the orthonormal basis of the coordinate
system. Suppose $[\bm{Q}]$ is the transformation matrix from principal
basis $\{\bm{n}_1,\bm{n}_2,\bm{n}_3\}$ to arbitrary basis $\{\bm{n}_1^{'},\bm{n}_2^{'},\bm{n}_3^{'}\}$, then:
$$
 [\tilde{\sigma}^{'}] = [\bm{Q}][\tilde{\sigma}] =
 [\bm{Q}][\bm{M}][\sigma] = [\bm{Q}][\bm{M}][\bm{Q}]^{-1}[\sigma^{'}] = [\bm{M}^{'}][\sigma^{'}] 
$$
$$
[\bm{M}^{'}] = [\bm{Q}][\bm{M}][\bm{Q}]^{-1}
$$

\subsection{Effective Strain Tensor}

The effective strain tensor is defined according to the hypothesis of
equivalence of the strain energy, i.e., the strain energy of the
damaged material is equivalent to the fictitious undamaged material:
$$
\epsilon = \frac{\partial{W(\sigma,\bm{D})}}{\partial{\sigma}}
         = \frac{\partial{W_o(\tilde{\sigma})}}{\partial{\sigma}}
         =
         \frac{\partial{W_o(\tilde{\sigma})}}{\partial{\tilde{\sigma}}}:
           \frac{\partial{\tilde{\sigma}}}{\partial{\sigma}}
         = \mathbb{M}^T(\bm{D}):\tilde{\epsilon}
$$
Hence,
$$
\tilde{\epsilon} = \mathbb{M}^{-T}(\bm{D}):\epsilon
$$

\section{The Thermodynamic Approach}

\subsection{Finite Deformation Theory}

The effective stress tensor and the effective strain tensor expressed
in the finite deformation theory are:
$$
\tilde{\bm{S}} = \mathbb{M}(\bm{D}):\bm{S}
$$
$$
\tilde{\bm{E}} = \mathbb{M}^{-T}(\bm{D}):\bm{E}
$$

$\bm{E}$ and $\bm{S}$ are respectively the Green-Lagrange strain
tensor and the second Piola-Kirchhoff stress tensor.

\subsection{Plasticity Model}

We employ the additive decomposition of the Green-Lagrange strain tensor
to describe plasticity in the field of large deformation:
$$
\bm{E}(\bm{X},t) = \bm{E}^e(\bm{X},t)+\bm{E}^p(\bm{X},t)
$$

\subsection{Helmholtz Free Energy}

The Helmholtz free energy $\Psi$ denotes the free energy per unit mass
in the material. We assume it consists of the free energy
$\Psi^{e}$ caused by the elastic deformation, the energy
$\Psi^{p}$ due to plastic deformation, and of the free energy
$\Psi^{\bm{D}}$ related to damage.
$$
\Psi(\bm{E}^e,\bm{D},\alpha,\beta) = \Psi^{e}(\bm{E}^e,\bm{D}) + \Psi^{p}(\alpha) + \Psi^{\bm{D}}(\beta)
$$

$\Psi^{p}(\alpha)$ and $ \Psi^{\bm{D}}(\beta)$ measures the
hardening (softening) effect due to plasticity and damage
respectively. $\alpha$ and $\beta$ are two scalars, $\alpha$ is the plastic hardening variable and $\beta$
is the damage strengthening variable. They can be omitted when no hardening
(softening) effect is presented, i.e., perfect plasticity
(damage). $\Psi^{e}(\bm{E}^e,\bm{D})$ is the general hyperelastic
constitutive model coupled with damage. As hyperelastic materials are
usually described with the energy density $W$ (per unit volume), it's
necessary to mention the relationship between the elastic part of
Helmholtz free energy $\Psi$ and the elastic strain
energy density $W$(per unit volume): $W = \rho\Psi$.

\subsection{The Clausius-Planck Inequality}

The plastic deformation and damage process are irreversible accompanied
with internal dissipation. From the second law of thermodynamics, the Clausius-Planck inequality in the reference configuration is given by:
$$
\bm{D}_{int} = \bm{S}:\dot{\bm{E}} - \rho_0[\eta\dot{T}+\dot{\Psi}] \geq 0
$$

We assume the whole process is isothermal, i.e., the temperature stays
fixed. The Clausius-Planck inequality becomes:
$$
\bm{D}_{int} = \bm{S}:\dot{\bm{E}} - \rho_0\dot{\Psi} \geq 0
$$

From the rate of change of $\Psi$,
$$
\dot{\Psi}=\frac{\partial\Psi}{\partial\bm{E}^e}:\dot{\bm{E}}^e+\frac{\partial\Psi}{\partial\bm{D}}:\dot{\bm{D}}
+\frac{\partial\Psi}{\partial\alpha}\cdot\dot{\alpha}+\frac{\partial\Psi}{\partial\beta}\cdot\dot{\beta}
$$

the internal dissipation becomes:

$$
\bm{D}_{int} = (\bm{S}-\rho_0\frac{\partial\Psi}{\partial\bm{E}^e}):\dot{\bm{E}}^e +\bm{S}:\dot{\bm{E}}^p- \rho_0(\frac{\partial\Psi}{\partial\bm{D}}:\dot{\bm{D}}
+\frac{\partial\Psi}{\partial\alpha}\cdot\dot{\alpha}+\frac{\partial\Psi}{\partial\beta}\cdot\dot{\beta}) \geq 0
$$

The above inequality holds for any loading/unloading process, e.g.,
one where $\dot{\bm{E}}^e=0$. Hence:
$$
\bm{S}=\rho_0\frac{\partial\Psi}{\partial\bm{E}^e}
$$

Introduce the generalized forces associated with the free variables:
$$
\bm{Y}=-\rho_0\frac{\partial\Psi}{\partial\bm{D}},\ A=-\rho_0\frac{\partial\Psi}{\partial\alpha},\ B=-\rho_0\frac{\partial\Psi}{\partial\beta} 
$$

The internal dissipation is expressed as:
$$
\bm{D}_{int} = \bm{S}:\dot{\bm{E}}^p+\bm{Y}:\dot{\bm{D}}+A\cdot\dot{\alpha}+B\cdot\dot{\beta}\geq 0
$$

\subsection{The Yield Criteria and Damage Criteria}

The yield criteria is defined as follows, considering only isotropic hardening:
$$
||\tilde{E}||-F^p(\bm{S},A,\bm{D})\leq 0
$$
$||\tilde{E}||$ is the norm of the effective stress tensor.

Similarly, the damage criteria is defined as:
$$
||Y||-F^D(\bm{Y},B)\leq 0
$$

\subsection{Evolution of Plasticity and Damage}

\bibliographystyle{eg-alpha}
\bibliography{ref}

\end{document}
